\documentclass{article}
\usepackage{hyperref}

\hypersetup{pdfnewwindow=true, pdfborder={0 0 0}}


\title{Geschiedenis van de Wetenschappen: Samenvatting}
\author{Midas Lambrichts}
\date{2016-2017}


\begin{document}

  \maketitle
  \tableofcontents
  \newpage

%%%%%%%%%%%%%%%%%%%%%%%%%%%%%%%%%%%%%%%%%%%%%%%%%%%%%%%%%%%%%%%%%%%%%%%%%%%%%%%%%%%%%%%%%%%%%%%%%%%%%%%%%%%%%%%%%%%%%%%%%%%%%%%%%%%%%%%%%%%%%%%%%%%%%%%%%%%%%%%%%%%%%%%%%%%%%%%%%%%%%%%%%%%%
  \section*{Voorwoord}
  \addcontentsline{toc}{section}{Voorwoord}
  Dit bestand is een samenvatting van de cursus van het vak ``Geschiedenis van de Wetenschappen [G0L57a]'' in het academiejaar 2016-2017. De cursus is op dit moment nog niet volledig en bijgevolg vat deze samenvatting ook niet alle leerstof samen. Het laatste hoofdstuk van de cursus is momenteel ``11 - Wetenschap aan de universiteit'', indien je deze samenvatting later bekijkt en de cursus ondertussen meer van de leerstof omvat, zal deze samenvatting dus logischerwijs niet meer heel de cursus omvatten.

  Deze samenvatting volgt de structuur van de cursus, dus soms worden dingen herhaald of worden er dingen gezegd om intrede in het volgende hoofdstuk te maken. Aangezien dit een samenvatting is wordt wel aangeraden om de cursus \'e\'en keer door te nemen.

  Als je deze samenvatting wilt uitbreiden of verbeteren mag je altijd een pull request sturen naar de repository op GitHub (\url{https://github.com/MidasLamb/GeschiedenisVanDeWetenschappen}). Gelieve wel dezelfde stijl van indentaties aan te houden zodat de structuur duidelijk blijft in het bron bestand.
  
  Ik maak ook gebruik van de Oxford Comma, ook als is dit niet correct in het Nederlands. De reden hiervoor is om mogelijke verwarring te vermijden.

%%%%%%%%%%%%%%%%%%%%%%%%%%%%%%%%%%%%%%%%%%%%%%%%%%%%%%%%%%%%%%%%%%%%%%%%%%%%%%%%%%%%%%%%%%%%%%%%%%%%%%%%%%%%%%%%%%%%%%%%%%%%%%%%%%%%%%%%%%%%%%%%%%%%%%%%%%%%%%%%%%%%%%%%%%%%%%%%%%%%%%%%%%%%
  \newpage
  \section*{Inleiding}
  \addcontentsline{toc}{section}{Inleiding}
    \begin{itemize}
      \item Moderne wetenschap:
      \begin{itemize}
        \item Ontstaan tussen 1500 en 1700
        \item ``Geboren'' in West-Europa (voortgebouwd op ontwikkelingen in andere culturen + specifieke Europose kenmerken)
        \item Na stabiele identiteit $\Rightarrow$ Uitgewaaierd
      \end{itemize}

      \item Tot 1500: Westen was zoals de rest:
      \begin{itemize}
        \item Periodes van Wetenschappelijke bedrijvigheid en periodes van verval.
        \item Wetenschap was beperkt tot kleine elite.
        \item Wetenschap stond in diesnt van politieke, religieuze of economische legitimatie.
        \item Wetenschap nooit van cruciaal belang in samenleving.
      \end{itemize}

      \item 1500-1599: Verandering:
      \begin{itemize}
        \item Itali\"e: ontstaan wetenschappelijke beweging $\Rightarrow$ verspreidde snel in Europa.
        \item Wetenschappelijke kennis neemt exponentieel toe.
        \item Wetenschap: ge\"institutionaliseerd in opleidingen, onderzoeksinstituten, en beroepsprofielen.
        \item Wetenschappelijke kennis nodig om deel te nemen aan publiek debat $\Rightarrow$ burger moet steeds meer ``Wetenschappelijk geletterd'' zijn.
      \end{itemize}

      \item Moderne wetenschap:
      \begin{itemize}
        \item Anders dan wat andere culturen hebben voortgebracht.
        \item Steunt op combinatie van:
        \begin{itemize}
          \item Observaties
          \item Experimenten
          \item Kwantitatieve modellen
          \item Abstract-theoretische begrippen
        \end{itemize}
        \item Kennis op grote schaal openbaar (publicaties, lezingen,...)
        \item Wetenschappelijke gemeenschap waakt over integriteit.
        \item Wetenschap $\leftrightarrow$ Technologie
        \item Het onstaan roept vragen op:
        \begin{itemize}
          \item Welke intellectuele, politieke, en technologische ontwikkelingen hebben de nieuwe wetenschap mogelijk gemaakt?
          \item In hoeverre waren deze ontwikkelingen al aanwezig?
          \item Op welke manier draagt moderne wetenschap een Europese stempel? Of is ze universeel?
        \end{itemize}
        \item Vragen kunnen in 3 categori\"en worden opgedeeld:
        \begin{itemize}
          \item Internalistisch: Enkel kijken naar de idee\"en.
          \item Externalistisch: Kijken naar de omgeving (vb. ontdekkingsreizen)
          \item Cultuurhistorisch: Wat geeft wetenschap zijn ``Autoriteit''? (Hier wordt de nadruk op gelegd in de rest!)
        \end{itemize}
      \end{itemize}
    \end{itemize}

%%%%%%%%%%%%%%%%%%%%%%%%%%%%%%%%%%%%%%%%%%%%%%%%%%%%%%%%%%%%%%%%%%%%%%%%%%%%%%%%%%%%%%%%%%%%%%%%%%%%%%%%%%%%%%%%%%%%%%%%%%%%%%%%%%%%%%%%%%%%%%%%%%%%%%%%%%%%%%%%%%%%%%%%%%%%%%%%%%%%%%%%%%%%
  \newpage
  \section{Het uitblijven van de Copernicaanse revolutie}
    \begin{itemize}
      \item 1543: 2 werken:
      \begin{itemize}
        \item De Revolutionibus Orbium Coelestium (Copernicus)
        \begin{itemize}
          \item Geocentrisme $\Rightarrow$ Heliocentrisme
          \item $\leftrightarrow$ Dominante wereldbeeld van Aristoteles dat aan Universiteiten heersde.
        \end{itemize}
        \item De Humani Corporis Fabrica (Vesalius)
        \begin{itemize}
          \item Eerste volledig anatomische beschrijving.
          \item Studie van gezond lichaam.
          \item Oproep tot hervorming van geneeskunde (Aansporing wetenschappelijke grondslagen te onderzoeken)
        \end{itemize}
      \end{itemize}
      \item Beide werken zijn historisch belangrijk, maar onmiddelijke impact op wetenschappelijk denken heel verschillend:
      \begin{itemize}
        \item De Humani Corporis Fabrica (Vesalius)
        \begin{itemize}
          \item Bestseller.
          \item Iedere universiteit volgde zijn voorbeeld.
        \end{itemize}
        \item De Revolutionibus Orbium Coelestium (Copernicus)
        \begin{itemize}
          \item Bijna onopgemerkt. (Maar 12 auteurs voor 1600 die denkwijze deelde.)
          \item Numerieke berekeningen en observaties werden overgenomen, maar fysische betekenis werd niet besproken.
          \item Geen Copernicaanse Revolutie in de 16\textsuperscript{de} eeuw geen sprake.
        \end{itemize}
      \end{itemize}
      \item Waarom bleef de Copernicaanse Revolutie uit? 
      \begin{itemize}
        \item Revolutionibus werd beschouwd als een vrijblijvende wiskundige hypothese, zonder gevolgen in de realiteit. (Komt door voorwoord buiten Copernicus' weten om).
        \item Copernicus maakte zich zorgen voor spottende reacties $\Rightarrow$ die bleven uit omdat het Heliocentrisme werd genegeerd.
        \item 
      \end{itemize}
      \item OPGEPAST: Niet denken vanuit moderne karakter van het werk, en de tijdsgenoten de schuld geven, maar het standpunt van de tijdsgenoten innemen en het werk toetsen.
      \begin{itemize}
        \item Wetenschapper werkt juist in die tijd $\Rightarrow$ definieert project op basis van begrippen/opvattingen van die tijd.
        \item Om dit te onderzoeken moeten we dus ook die wetenschap uit die tijd begrijpen:
        \begin{itemize}
          \item In die tijd was Astronoom (wetenschap nu)=Astroloog (spekkers met de dierenriem): Astronomie stond enkel in teken van de astrologie.
          \item Copernicus kwam in contact met de ``Almagest'' (astronomisch standaardwerk). Dit gebruikte cirkels met aarde in midden. Copernicus plaatste de zon in het midden en dit vereenvoudigde alles. Bovendien kon hij hiermee de volgorde van de planeten bepalen.
          \item Revolutionibus: geen Astrologisch werk $\Rightarrow$ Werd wel beschouwd in hoe het tot debatten over Astrologie kon bijdragen. (Waren veel boeken die in debatten dienst deden.) $\Rightarrow$ Het boek ging alleen over Astronomie, dus niet zo nuttig in debatten. $\Rightarrow$ Astronomie werd overschaduwd door Astrologie.
          \item OPGELET: Bovenstaande steunt op Westman $\Rightarrow$ Kan zijn dat Copernicus zich probeerde te distanti\"eren van Astrologie.
        \end{itemize}
      \end{itemize}
      \item De Copernicaanse Revolutie bleef dus uit, maar het heliocentrisme moest wel ingang vinden in andere subcultuur (dan Astrologie/Astronomie), namelijk de natuurfilosofie.
    \end{itemize}

%%%%%%%%%%%%%%%%%%%%%%%%%%%%%%%%%%%%%%%%%%%%%%%%%%%%%%%%%%%%%%%%%%%%%%%%%%%%%%%%%%%%%%%%%%%%%%%%%%%%%%%%%%%%%%%%%%%%%%%%%%%%%%%%%%%%%%%%%%%%%%%%%%%%%%%%%%%%%%%%%%%%%%%%%%%%%%%%%%%%%%%%%%%%
  \newpage
  \section{Een wereld op papier}
    \begin{itemize}
      \item Niewigheid in 16\textsuperscript{de} eeuwse wetenschap: ILLUSTRATIES
      \begin{itemize}
        \item Maakte het mogelijk voor wetenschappers die geen toegang hadden tot dingen (zoals Neushoorns), om deze toch op een (min of meer) natuurgetrouwe manier te bekijken/te onderzoeken.
      \end{itemize}
      \item 16\textsuperscript{de} eeuwse natuurgetrouwe illustraties $\leftrightarrow$ Middeleeuwse schematische voorstellingen.
      \item Innovatie lijkt gevolg van empirische interesse
      \item Maar er was ook veel tegenstand
      \begin{itemize}
        \item Maakte directe observatie onnodig (waarom echt dissecteren als afbeelding toch essenti\"ele informatie bevat?)
        \item Vragen bij betrouwbaarheid van afbeeldingen (Vesalius' spiermannen waren idealen, geen afbeeldingen van dissecties, afbeeldingen van planten in verschillende groeistadia tegelijk)
        \item Kaarten konden niet gemakkelijk worden aangepast aan nieuwe ontdekkingen.
        \item Uitgevers betaalden voor afbeeldingen $\Rightarrow$ bepaalde wat in boek kwam (dus commerci\"ele redenen: monsters of mythologische figuren, die niks met tekst te maken hadden)
      \end{itemize}
      \item Afbeeldingen wijzen op verschuiving in het wetenschappelijk denken:
      \begin{itemize}
        \item Afbeelding toont particulier object $\leftrightarrow$ Universele begrippen (Aristoteles' essentie van de dingen)
      \end{itemize}
      \item Ook aandacht verschuift:
      \begin{itemize}
        \item Gericht op veelheid van planten en dieren, zonder poging om ordening/logica aan te brengen.
        \item Klein onderscheid tussen 2 planten kan van groot belang zijn $\leftrightarrow$ Universitaire ideaal (planten enkel voor geneeskunde) $\Rightarrow$ Natuurlijke historie vooral ontwikkeld buiten Universiteit, in netwerk van correspondenten.
        \item $\Rightarrow$ Nieuwe subcultuur van natuuronderzoekers: gedreven door praktische nieuwsgierigheid en niet door theoretische ambities.
      \end{itemize}
      \item Gebruik van afbeeldingen suggereert een tweede belangrijke verschuiving:
      \begin{itemize}
        \item Afbeeldingen vaak gebruikt in technische handleidingen, nodig om te begrijpen wat er in de tekst bedoeld wordt.
        \item $\Rightarrow$ Vervaging grens natuurlijk-artificieel: Afbeelding altijd artificieel, maar konden toch essentie vatten ($\leftrightarrow$ Aristoteles: artificieel (experimenten) zijn niet goed.)
      \end{itemize}
      \item Visuele cultuur is weerspiegeling van de grote interesse in natuurverschijnselen. (Zoals te zien in grote verzamelingen die niet wetenschappelijk zijn samengesteld, maar wel wetenschappers uitnodigde.)
    \end{itemize}

%%%%%%%%%%%%%%%%%%%%%%%%%%%%%%%%%%%%%%%%%%%%%%%%%%%%%%%%%%%%%%%%%%%%%%%%%%%%%%%%%%%%%%%%%%%%%%%%%%%%%%%%%%%%%%%%%%%%%%%%%%%%%%%%%%%%%%%%%%%%%%%%%%%%%%%%%%%%%%%%%%%%%%%%%%%%%%%%%%%%%%%%%%%%
  \newpage
  \section{Subculturen van wetenschap}
    \begin{itemize}
      \item 16\textsuperscript{de} eeuw:
      \begin{itemize}
        \item Ontstaan verschillende wetenschappelijke subculturen, gebaseerd op eigen visie natuurkennis.
        \item Onderscheiden van universiteiten. (Nauwelijk interesse in natuurwetenschap: Aristotelische natuurfilosofie in teken van academische afstudeerrichtingen (theologie, kerkelijk/burgelijk recht, geneeskunde))
        \item Universiteiten waren niet conservatief $\Rightarrow$ Open instellingen met informele discussies/idee\"enuitwisseling studenten-professoren.
        \item Universiteit: knooppunt internationaal verkeer idee\"en, boeken, objecten. Maar toch discours gebaseerd op Aristoteles' natuurfilosofie.
      \end{itemize}
      \item Subculteren:
      \begin{itemize}
        \item Humanisten (Desiderius Erasmus, Thomas More):
        \begin{itemize}
          \item Gebaseerd op filologische studie van teksten uit de Oudheid.
          \item Discussies over authenticiteit van overgeleverde teksten, interpretaties van Griekse/Latijnse uitdrukkingen.
          \item Uiteindelijke logische coherentie was GEEN doel! $\Rightarrow$ Enkel belangrijk: getrouwe weergave van de Ouden.
          \item Vorm en Stijl heel belangrijk!
          \item Toch brachten ze origineel werk (Vesalius weerlegde Gelanus terwijl hij er op steunde.)
        \end{itemize}
        \item Natuurlijke Historie:
        \begin{itemize}
          \item Studie van planten, dieren, en gesteenten.
          \item Ontstaan vanuit Humanisme.
          \item Probeerden planten die Dioscorides (Oude) had beschreven opnieuw te identificeren.
          \item Cre\"eerde internationaal netwerk van corresponderende Botanici.
          \item Ge\"ilustreerde kruidenboeken.
          \item Werd onderdeel van de elitaire cultuur van rijke verzamelaars.
          \item Cultuur verbrokkelde begin 17\textsuperscript{de} eeuw:
          \begin{itemize}
            \item Botanici voelden nood aan systematisering en classificatie.
            \item Verzamelaars hadde noog voor exotisch/esthetisch.
            \item Belangen van beide groepen (die samen Natuurhistorici vormden) liepen meer en meer uiteen.
            \item Het aantal te beschrijven soorten was ook gigantische toegenomen.
          \end{itemize}
        \end{itemize}
        \item Wonderboeken:
        \begin{itemize}
          \item Boeken waarin medische en technische recepten werden uitgelegd aan de ``gewone man'' in volkstaal.
          \item Commerci\"eel $\Rightarrow$ gaven goed aan wat publiek van wetenschap verwachtte.
          \item Gebaseerd op lokale ambachtslui.
          \item Auteurs waren meestal Humanisten.
          \item Gaf kritiek op academische wetenschap (boekverbranding) en riep op om kennis bij lokale ambachtsliederen te zoeken.
          \item Interesse voor Hermitisme:
          \begin{itemize}
            \item Verzamelnaam voor filosofisch-religieuze stromingen voor het Occulte.
            \item Inspiratie bij Alchemie, Astrologie, en Natuurlijke Magie.
          \end{itemize}
          \item Brede interesse voor wonderboeken $\Rightarrow$ Brede intersse voor nieuw wetenschappelijk kennisdomein: die van de ambachten.
          \item Introduceerde gebruik van doelbewuste experimenten met systematisch rapporteren.
          \item Nadruk op praktische kennis, met impliciete conclusie dat de natuur op zo'n manier verstaan moet worden.
          \item Moderen Wetenschap distantieerde zich achteraf wegens publicatie-cultuur en verwerping esoterische elementen van kennis over de natuur.
        \end{itemize}
        \item Wiskundige Cultuur:
        \begin{itemize}
          \item Kunstenaars (perspectiefleer) en ingenieurs (architectuur, cartografie,...)
          \item In Itali\"e
          \item Situeert zich tussen Universitair (Astrologie/Astronomie) en praktische reken- en meetkunde.
          \item Wiskunde als toe-eigening van status: door wiskunde kreeg kunst een theoretische legitimatie.
          \item Kunstenaars vooral tewerkgesteld in adellijke hoven als hofwiskundigen.
          \item Nadruk op uistallen van pracht, rijkdom, en status.
          \item Bracht Galilei voort.
          \item Ook ten Noorden van de Alpen: andere achtergrond:
          \begin{itemize}
            \item Nadruk op praktisch nut van ingenieurs in dienst van handelaars/overheden (Engeland: Navigatie -Nederland: Inpoldering/Vestingbouw).
          \end{itemize} 
        \end{itemize}
      \end{itemize}
    \end{itemize}

%%%%%%%%%%%%%%%%%%%%%%%%%%%%%%%%%%%%%%%%%%%%%%%%%%%%%%%%%%%%%%%%%%%%%%%%%%%%%%%%%%%%%%%%%%%%%%%%%%%%%%%%%%%%%%%%%%%%%%%%%%%%%%%%%%%%%%%%%%%%%%%%%%%%%%%%%%%%%%%%%%%%%%%%%%%%%%%%%%%%%%%%%%%%
  \newpage
  \section{De wiskundige natuurfilosofie van Galilei}
    \begin{itemize}
      \item Tot 1610:
      \begin{itemize}
        \item Moeizaam!
        \item Gaf geneeskunde op voor wiskunde.
        \item Eerste job: Perspectiefleer op Florentijnse kunstacademie
        \item 1589: Aangesteld tot hoogleraar wiskunde (Universiteit Pisa) $\Rightarrow$ Verruilde voor aanstelling aan universiteit van Padua.
        \item Hoogleraar wiskunde verdiende niet goed $\Rightarrow$ Bijverdienen door maken van wiskundige instrumenten (kompas,...)
        \item Kritisch over natuurfilosofische bewegegingsleer van Aristoteles (maar maakte weinig indruk).
      \end{itemize}
      \item 1609:
      \begin{itemize}
        \item Hoorde van uitvinding in Nederland: telescoop.
        \begin{itemize}
          \item Bouwde er zelf en verkocht aan Venetiaanse Senaat
          \item Keek zelf naar de sterren $\Rightarrow$ deed ontdekkingen:
          \begin{itemize}
            \item Meer sterren dan met het blote oog zichtbaar.
            \item Maan heeft oneffen oppervlak.
            \item Rond Jupiter bewegen 4 sattelieten.
          \end{itemize}
          \item Ieder van deze ontdekkingen had gevolgen voor Natuurfilosofie:
          \begin{itemize}
            \item Waarom heeft God sterren geschapen die we niet met het blote oog kunnen zien?
            \item Waarin verschilt de Maan nu nog van de Aarde, aangezien de opervlakken allebei ruw zijn?
            \item Waarom is de aarde middelpunt, aangezien er manen rond Jupiter zijn (en dus niet rond de aarde cirkelen)?
          \end{itemize}
        \end{itemize}
        \item Galilei publiceerde bevindingen in een kleine brochure $\Rightarrow$ Grote sensatie:
        \begin{itemize}
          \item Is wat Galilei gezien heeft wel echt? Of gezichtsbedrog door fouten in lenzensysteem?
          \item Florentijnse groothertog stelde hem aan tot hofmathematicus en -filosoof.
        \end{itemize}
      \end{itemize}
      \item Galilei interpreteerde ontdekkingen als bewijzen voor Heliocentrisme
      \begin{itemize}
        \item Kritiek op traditionele natuurfilosofie werd gebruikt om autoriteit van de Kerk te ondermijnen $\Rightarrow$ Censuur op boek van Copernicus.
        \begin{itemize}
          \item Verboden Heliocentrisme als waar te verdedigen (wel als hypothese onderzoeken)
          \item Galilei hield zich aan dit verbod.
        \end{itemize}
      \end{itemize}
      \item  ``Dialogo sopra i due massimi sistemi del mondo'' (1632)
      \begin{itemize}
        \item Boek uit dat discussie over het wereldbeeld tot onderwerp had
        \item Galilei werd naar Rome geroepen om zich te verdedigen.
        \item $\Rightarrow$ 1633: Veroordeeld theorie\"en af te zweren + huisarrest
      \end{itemize}
      \item ``Discorsi e Dimostrazioni Matematiche, intorno a due nuove scienze'' (1638)
      \begin{itemize}
        \item 2 nieuwe wetenschappen: Kinematica + sterkteleer
      \end{itemize}
      \item Hofwiskundige:
      \begin{itemize}
        \item Galilei gebruikte polemische, argumentatieve stijl.
        \item MAAR: manier waarop hij wiskundige argumenten gebruikte om fysische conclusies te onderbouwen was ongewoon (zelfs irrelevant):
        \begin{itemize}
          \item Wiskunde was een geheel van disciplines die meetbare aspecten bestuderen, maar geen uitspraak konden doen over essentie/oorzaken.
          \item Galilei vond van wel.
        \end{itemize}
      \end{itemize}
      \item Wiskundige Natuurfilosofie:
      \begin{itemize}
        \item Galilei was \'e\'en van de eerste verdedigers: Boek van de natuur was geschreven in het alfabet van de Wiskunde.
        \item Uit studie van wiskundige relaties kon natuur begrepen worden.
        \item Bondgenoten in Jezu\"iten maar deze wouden grens Wiskunde-Natuurfilosofie behouden. $\Rightarrow$ Jezu\"iten werden tegenstanders van Galilei.
      \end{itemize}
      \item Invloed Galilei bleef beperkt tot Itali\"e.
      \begin{itemize}
        \item Publiceerde alleen in het Italiaans.
        \item Weigerde te corresponderen met buitenlanders.
      \end{itemize}
      \item Toch was hij buiten Itali\"e een bekend figuur geworden:
      \begin{itemize}
        \item Vooral door zijn aanvol op Aristotelische natuurfilosofie en verdediging van Copernicaanse wereldbeeld.
        \item Info verspreidde zich via informele netwerken (R\'epublique des Lettres)
        \item Diodati schreef een verdediging van Galilei $\Rightarrow$ Beeld van Galilei als opstandige geleerde die Aristotelische natuurfilosofie uitdaagde.
        \item Probleem was dat Galilei geen alternatief had voor de Aristotelische natuurfilosofie.
      \end{itemize}
      \item Mersenne:
      \begin{itemize}
        \item Netwerk van correspondenten met meer focus op wetenschap.
        \item Maakte onderscheid tussen uiterlijke fenomenen (wiskundig/experimenteel) en inwendige eigenschappen (enkel via hypotheses)
        \item Vond Galilei's bewegingsleer perfect voor uiterlijke fenomenen.
        \item Cre\"eerde debat over ``Mechanistische natuurwetenschap''
        \item Voor Mersenne niet in tegenspraak met Aristotelische natuurfilosofie, want het ging enkel over het uitwendige!
        \item Ren\'e Descartes zat ook in het netwerk, net zoals Galilei zou hij hem helpen verspreiden (maar met grotere impact dan de bedoeling!)
      \end{itemize}
    \end{itemize}

%%%%%%%%%%%%%%%%%%%%%%%%%%%%%%%%%%%%%%%%%%%%%%%%%%%%%%%%%%%%%%%%%%%%%%%%%%%%%%%%%%%%%%%%%%%%%%%%%%%%%%%%%%%%%%%%%%%%%%%%%%%%%%%%%%%%%%%%%%%%%%%%%%%%%%%%%%%%%%%%%%%%%%%%%%%%%%%%%%%%%%%%%%%%
  \newpage
  \section{Een nieuwe kosmologie}
    \begin{itemize}
      \item Scholastieke natuurfilosofie was niet conservatief en onveranderlijk
      \begin{itemize}
        \item Basis was Aristotelische natuurfilosofie
        \item Binnen dat kader grote vrijheid:
        \begin{itemize}
          \item Aristoteles werd aangevuld met hoofdstukken over geografie, hydrografie,...
          \item $\Rightarrow$ Natuurfilosofie kreeg meer nadruk in het geheel van Aristoteles' werk (in verhouding met logica en metafysica)
          \item $\Rightarrow$ Afstand met oorspronkelijke teksten werd steeds groter.
          \item Betekende niet dat Aristotelische kader opzij kon worden geschoven.
          \begin{itemize}
            \item Wie dat wel deed werd als ketter beschouwd.
            \item Aristotelische natuurfilosofie was begrippenkader voor Christelijke natuurbeschouwing, wat fundament van de maatschappelijke orde was.
            \item $\Rightarrow$ Oplossing: Een formele scheiding tussen wiskundige wetenschappen en natuurfilosofie (zoals Mersenne)
            %\begin{itemize}
              \item Wiskundige wetenschappen ontwikkelde zich vooral in sociale milieus die universitaire onderwijswereled niet voor de voeten liep.
            %\end{itemize}
          \end{itemize}
        \end{itemize}
      \end{itemize}
      \item Ren\'e Descartes
      \begin{itemize}
        \item Universitaire opleiding maar geen aanstelling.
        \item Wiskunde de sleutel tot zekere kennis.
        \begin{itemize}
          \item Niet enkel uitwendige aspecten, maar ook interne/verborgen oorzaken.
          \item Kern was gelijkstelling ruimte en materie (Ruimte wordt altijd door iets opgevuld)
          \item Interactie van alle deeltjes kon mechanisch worden geanalyseerd.
        \end{itemize}
        \item Zijn systeem verklaarde alle natuurverschijnselen (maar enkel Kwalitatief).
        \item Vernam van Galilei zijn veroordeling en zette zijn werk stop uit voorzorg.
        \item Publiceerde minder ambitieus boek: ``Discours de la méthode''
        \begin{itemize}
          \item Manier om complexe problemen op te lossen: Probleem ontleden, onderdelen afzonderlijk oplossen, van afzonderlijke probleem terug op bouwen.
          \item Maakte grote indruk $\Rightarrow$ uitgebreide correspondentie (ook over metafysische grondslagen van filosofie + Godsbegrip in mechanische wereld)
        \end{itemize}
        \item ``Principia Philosophiae''
        \begin{itemize}
          \item Uitwerking Natuurfilosofie.
          \item Begin van invloed op Universiteiten.
          \item Ook mikpunt van hevige kritiek
        \end{itemize}
        \item Bij zijn overlijden had ich een groep Cartesianen gevormd, die zijn denkbeelden verder verspreidde.
        \item Aantrekkelijkheid van zijn natuurfilosofie:
        \begin{itemize}
          \item Einde aan de ingewikkelde en abstracte categorie\"en.
          \item Wiskundige grondslag in zijn natuurfilosofie verbond theoretische natuurfilosofie met de praktische disciplines.
          \item Systeem was compact en allesomvattend.
        \end{itemize}
        \item Veel Cartesianen namen mechanische wereldbeeld over zonder metafysische ballast.
        \item Wiskunde kwam in zijn systeem nauwelijks van pas (wegens kwalitatieve uitwerking.)
        \item Gaf zelf aan dat zijn wereldsysteem niet absoluut zeker is.
        \item Bewijs kon enkel a posteriori
        \begin{itemize}
          \item Opende de weg naar experimenten als toets voor theorie\"en.
          \item Zorgde ervoor dat Certesianisme een relatief open systeem bleef, want hypothesen waren nooit helemaal bewezen $\Rightarrow$ Konden steeds worden aangepast.
        \end{itemize}
      \end{itemize}
      \item Nog voor de 17\textsuperscript{de} eeuw was Aristoteles onttroond als enige autoriteit aan de universiteiten.
      \item Vanuit informele netwerken binnengedrongen in academische wereld!
      \item Cartesianisme had vooral invloed op geneeskunde.
    \end{itemize}

%%%%%%%%%%%%%%%%%%%%%%%%%%%%%%%%%%%%%%%%%%%%%%%%%%%%%%%%%%%%%%%%%%%%%%%%%%%%%%%%%%%%%%%%%%%%%%%%%%%%%%%%%%%%%%%%%%%%%%%%%%%%%%%%%%%%%%%%%%%%%%%%%%%%%%%%%%%%%%%%%%%%%%%%%%%%%%%%%%%%%%%%%%%%
  \newpage
  \section{Nieuwsgierigheid en materi\"ele cultuur}
    \begin{itemize}
      \item Internalistische geschiedschrijving:
      \begin{itemize}
        \item Verdwijnen van Aristotelische natuurfilosofie en introductie Cartesiaanse theorie\"en is een logische stap in het proces van mechanisering van het wereldbeeld.
        \item Deze visie gaat voorbij aan grote verschillen tussen wetenschapsopvattingen in verschillende culturen.
      \end{itemize}
      \item Descartes:
      \begin{itemize}
        \item Maar matig ge\"interesseerd in Galilei
        \begin{itemize}
          \item Bewonderde de wiskundige redeneringen
          \item Vond het niks dat Galilei geen verklaring voor zijn natuurwetten had. (Was volgens Descartes juist de taak van natuurfilosofie)
          \item Maakte geen gebruik van Galilei's bevindingen in zijn eigen Principia
        \end{itemize}
        \item Verspreiding van zijn opvattingen werden gekleurd door de verwachtingen en interesses van zijn volgelingen:
        \begin{itemize}
          \item Strenge, op wiskundige axiomatieke opbouw van Descartes' Kosmologie werd door latere Cartesianen achterwege gelaten.
          \item Latere Cartesianen namen Descartes' mechanische kosmologie als uitgangspunt, maar legden nadruk op verklaring van concrete verschijnselen.
          \item GEVOLG: Cartesianisme werd gereduceerd tot een vaag, algemeen systeem, zonder aanspraak te maken op absolute zekerheid.
        \end{itemize}
      \end{itemize}
      \item Afgezwakt Cartesianisme:
      \begin{itemize}
        \item Noodzakelijke aanpassing aan de politieke en religieuze realiteit.
        \item In Cartesianisme was zeer weinig plek voor God (enkel Schepping).
        \item Descartes hanteerde strikte scheiding ziel (immatrieel) en lichaam (zuiver mechanisch). (Maar kon geen antwoord geven op hoe ze interageren.)
        \begin{itemize}
          \item Deze opvatting stuitte op veel verzet.
          \item Cartesianisme aan universiteiten en in populaire literatuur hield zich weg van metafysische aspect.
        \end{itemize}
      \end{itemize}
      \item Verschuiving Cartesianisme van rationalisme naar empirisme
      \begin{itemize}
        \item Zowel in evolutie van begrip wetenschap als van de maatschappelijke omgeving.
        \begin{itemize}
          \item Fascinatie voor het particuliere (curiosa).
        \end{itemize}
      \end{itemize}
      \item Francis Bacon:
      \begin{itemize}
        \item Natuur bestuderen door aanleggen inventarislijsten van alle natuurlijke verschijnselen (dus ook ongewone).
        \item Visie was basis voor genootschappen die basis vormden voor een wetenschappelijke subcultuur.
      \end{itemize}
      \item Ook elders werden er venootschapen opgericht.
      \begin{itemize}
        \item Werd vaak geeist dat men zich onthield van metafysische, religieuze, politieke, of speculatieve discussie.
      \end{itemize}
      \item Cultuur van Nieuwsgierigheid
      \begin{itemize}
        \item Aandacht voor nieuwe ontdekkingen overstijgt nadruk op theoretische verklaringen.
        \item Uiten zich door experimentele fysica aan universiteiten.
        \item Ook aanwezig in Cartesianen na Descartes!
        \begin{itemize}
          \item Komt door onvrede over gebrek effectiviteit geneeskunde.
          \item Artsen moesten autoriteit bevestigen.
          \begin{itemize}
            \item Wetenschappelijke basis versterken.
            \item Gebruik remedies lokale dokters.
            \item Maakte gebruik van Descartes' bloedsomloop.
            \begin{itemize}
              \item Lichaam was een machine!
            \end{itemize}
          \end{itemize}
        \end{itemize}
      \end{itemize}
      \item Interesse voor microscopie
      \begin{itemize}
        \item Artsen begonnen microscoop te gebruiken om interne structuur te beschrijven
      \end{itemize}
      \item Mechanisering van Descartes $\Leftrightarrow$ Mathematisering van Galilei
      \begin{itemize}
        \item Mechanisering:
        \begin{itemize}
          \item Onderzoek van objecten, met doel de mechanische oorzaken te achterhalen.
          \item Gelijkend op een klok: Natuur is een klok; Wetenschap moet dezelfde klok bouwen! (verwerven hierdoor kennis, ook al zijn we niet zeker)
        \end{itemize}
      \end{itemize}
    \end{itemize}

%%%%%%%%%%%%%%%%%%%%%%%%%%%%%%%%%%%%%%%%%%%%%%%%%%%%%%%%%%%%%%%%%%%%%%%%%%%%%%%%%%%%%%%%%%%%%%%%%%%%%%%%%%%%%%%%%%%%%%%%%%%%%%%%%%%%%%%%%%%%%%%%%%%%%%%%%%%%%%%%%%%%%%%%%%%%%%%%%%%%%%%%%%%%
  \newpage
  \section{Newton en de Wetenschappelijke Revolutie}
    \begin{itemize}
      \item Newton was bekroning Wetenschappelijke Revolutie: formuleerde expliciet doel en mothode van de experimentele wetenschappen.
      \item Newton's werk grijpt terug op oudere vormen van wetenschap.
      \begin{itemize}
        \item Wetenschapsgeschiedenis: belangrijk onderscheid te maken tussen:
        \begin{itemize}
          \item Beeld dat latere generaties van Newton geschetst hebben.
          \item Historische context waarin zijn werk tot stand kwam.
        \end{itemize}
      \end{itemize}
    \end{itemize}

%%%%%%%%%%%%%%%%%%%%%%%%%%%%%%%%%%%%%%%%%%%%%%%%%%%%%%%%%%%%%%%%%%%%%%%%%%%%%%%%%%%%%%%%%%%%%%%%%%%%%%%%%%%%%%%%%%%%%%%%%%%%%%%%%%%%%%%%%%%%%%%%%%%%%%%%%%%%%%%%%%%%%%%%%%%%%%%%%%%%%%%%%%%%
  \newpage
  \section{Iedereen Newtoniaan!}

%%%%%%%%%%%%%%%%%%%%%%%%%%%%%%%%%%%%%%%%%%%%%%%%%%%%%%%%%%%%%%%%%%%%%%%%%%%%%%%%%%%%%%%%%%%%%%%%%%%%%%%%%%%%%%%%%%%%%%%%%%%%%%%%%%%%%%%%%%%%%%%%%%%%%%%%%%%%%%%%%%%%%%%%%%%%%%%%%%%%%%%%%%%%
  \newpage
  \section{Nuttige wetenschap}

%%%%%%%%%%%%%%%%%%%%%%%%%%%%%%%%%%%%%%%%%%%%%%%%%%%%%%%%%%%%%%%%%%%%%%%%%%%%%%%%%%%%%%%%%%%%%%%%%%%%%%%%%%%%%%%%%%%%%%%%%%%%%%%%%%%%%%%%%%%%%%%%%%%%%%%%%%%%%%%%%%%%%%%%%%%%%%%%%%%%%%%%%%%%
  \newpage
  \section{Gescheiden wegen}

%%%%%%%%%%%%%%%%%%%%%%%%%%%%%%%%%%%%%%%%%%%%%%%%%%%%%%%%%%%%%%%%%%%%%%%%%%%%%%%%%%%%%%%%%%%%%%%%%%%%%%%%%%%%%%%%%%%%%%%%%%%%%%%%%%%%%%%%%%%%%%%%%%%%%%%%%%%%%%%%%%%%%%%%%%%%%%%%%%%%%%%%%%%%
  \newpage
  \section{Wetenschap aan de universiteit}


\end{document}
