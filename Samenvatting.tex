\documentclass{article}
\usepackage{hyperref}

\hypersetup{pdfnewwindow=true, pdfborder={0 0 0}}


\title{Geschiedenis van de Wetenschappen: Samenvatting}
\author{Midas Lambrichts}
\date{2016-2017}


\begin{document}

  \maketitle
  \tableofcontents
  \newpage

%%%%%%%%%%%%%%%%%%%%%%%%%%%%%%%%%%%%%%%%%%%%%%%%%%%%%%%%%%%%%%%%%%%%%%%%%%%%%%%%%%%%%%%%%%%%%%%%%%%%%%%%%%%%%%%%%%%%%%%%%%%%%%%%%%%%%%%%%%%%%%%%%%%%%%%%%%%%%%%%%%%%%%%%%%%%%%%%%%%%%%%%%%%%
  \section*{Voorwoord}
  \addcontentsline{toc}{section}{Voorwoord}
  Dit bestand is een samenvatting van de cursus van het vak ``Geschiedenis van de Wetenschappen [G0L57a]'' in het academiejaar 2016-2017. De cursus is op dit moment nog niet volledig en bijgevolg vat deze samenvatting ook niet alle leerstof samen. Het laatste hoofdstuk van de cursus is momenteel ``11 - Wetenschap aan de universiteit'', indien je deze samenvatting later bekijkt en de cursus ondertussen meer van de leerstof omvat, zal deze samenvatting dus logischerwijs niet meer heel de cursus omvatten.

  Deze samenvatting volgt de structuur van de cursus, dus soms worden dingen herhaald of worden er dingen gezegd om intrede in het volgende hoofdstuk te maken. Aangezien dit een samenvatting is wordt wel aangeraden om de cursus \'e\'en keer door te nemen.

  Als je deze samenvatting wilt uitbreiden of verbeteren mag je altijd een pull request sturen naar de repository op GitHub (\url{https://github.com/MidasLamb/GeschiedenisVanDeWetenschappen}). Gelieve wel dezelfde stijl van indentaties aan te houden zodat de structuur duidelij blijft in het bron bestand.

%%%%%%%%%%%%%%%%%%%%%%%%%%%%%%%%%%%%%%%%%%%%%%%%%%%%%%%%%%%%%%%%%%%%%%%%%%%%%%%%%%%%%%%%%%%%%%%%%%%%%%%%%%%%%%%%%%%%%%%%%%%%%%%%%%%%%%%%%%%%%%%%%%%%%%%%%%%%%%%%%%%%%%%%%%%%%%%%%%%%%%%%%%%%
  \newpage
  \section*{Inleiding}
  \addcontentsline{toc}{section}{Inleiding}

    \begin{itemize}
      \item Moderne wetenschap:
      \begin{itemize}
        \item Ontstaan tussen 1500 en 1700
        \item ``Geboren'' in West-Europa (voortgebouwd op ontwikkelingen in andere culturen + specifieke Europose kenmerken)
        \item Na stabiele identiteit $\Rightarrow$ Uitgewaaierd
      \end{itemize}

      \item Tot 1500: Westen was zoals de rest:
      \begin{itemize}
        \item Periodes van Wetenschappelijke bedrijvigheid en periodes van verval.
        \item Wetenschap was beperkt tot kleine elite.
        \item Wetenschap stond in diesnt van politieke, religieuze of economische legitimatie.
        \item Wetenschap nooit van cruciaal belang in samenleving.
      \end{itemize}

      \item 1500-1599: Verandering:
      \begin{itemize}
        \item Itali\"e: ontstaan wetenschappelijke beweging $\Rightarrow$ verspreidde snel in Europa.
        \item Wetenschappelijke kennis neemt exponentieel toe.
        \item Wetenschap: ge\"institutionaliseerd in opleidingen, onderzoeksinstituten, en beroepsprofielen.
        \item Wetenschappelijke kennis nodig om deel te nemen aan publiek debat $\Rightarrow$ burger moet steeds meer ``Wetenschappelijk geletterd'' zijn.
      \end{itemize}

      \item Moderne wetenschap:
      \begin{itemize}
        \item Anders dan wat andere culturen hebben voortgebracht.
        \item Steunt op combinatie van:
        \begin{itemize}
          \item Observaties
          \item Experimenten
          \item Kwantitatieve modellen
          \item Abstract-theoretische begrippen
        \end{itemize}
        \item Kennis op grote schaal openbaar (publicaties, lezingen,...)
        \item Wetenschappelijke gemeenschap waakt over integriteit.
        \item Wetenschap $\leftrightarrow$ Technologie
        \item Het onstaan roept vragen op:
        \begin{itemize}
          \item Welke intellectuele, politieke, en technologische ontwikkelingen hebben de nieuwe wetenschap mogelijk gemaakt?
          \item In hoeverre waren deze ontwikkelingen al aanwezig?
          \item Op welke manier draagt moderne wetenschap een Europese stempel? Of is ze universeel?
        \end{itemize}
        \item Vragen kunnen in 3 categori\"en worden opgedeeld:
        \begin{itemize}
          \item Internalistisch: Enkel kijken naar de idee\"en.
          \item Externalistisch: Kijken naar de omgeving (vb. ontdekkingsreizen)
          \item Cultuurhistorisch: Wat geeft wetenschap zijn ``Autoriteit''? (Hier wordt de nadruk op gelegd in de rest!)
        \end{itemize}
      \end{itemize}
    \end{itemize}

%%%%%%%%%%%%%%%%%%%%%%%%%%%%%%%%%%%%%%%%%%%%%%%%%%%%%%%%%%%%%%%%%%%%%%%%%%%%%%%%%%%%%%%%%%%%%%%%%%%%%%%%%%%%%%%%%%%%%%%%%%%%%%%%%%%%%%%%%%%%%%%%%%%%%%%%%%%%%%%%%%%%%%%%%%%%%%%%%%%%%%%%%%%%
  \newpage
  \section{Het uitblijven van de Copernicaanse revolutie}
    \begin{itemize}
      \item 1543: 2 werken:
      \begin{itemize}
        \item De Revolutionibus Orbium Coelestium (Copernicus)
        \begin{itemize}
          \item Geocentrisme $\Rightarrow$ Heliocentrisme
          \item $\leftrightarrow$ Dominante wereldbeeld van Aristoteles dat aan Universiteiten heersde.
        \end{itemize}
        \item De Humani Corporis Fabrica (Vesalius)
        \begin{itemize}
          \item Eerste volledig anatomische beschrijving.
          \item Studie van gezond lichaam.
          \item Oproep tot hervorming van geneeskunde (Aansporing wetenschappelijke grondslagen te onderzoeken)
        \end{itemize}
      \end{itemize}
      \item Beide werken zijn historisch belangrijk, maar onmiddelijke impact op wetenschappelijk denken heel verschillend:
      \begin{itemize}
        \item De Humani Corporis Fabrica (Vesalius)
        \begin{itemize}
          \item Bestseller.
          \item Iedere universiteit volgde zijn voorbeeld.
        \end{itemize}
        \item De Revolutionibus Orbium Coelestium (Copernicus)
        \begin{itemize}
          \item Bijna onopgemerkt. (Maar 12 auteurs voor 1600 die denkwijze deelde.)
          \item Numerieke berekeningen en observaties werden overgenomen, maar fysische betekenis werd niet besproken.
          \item Geen Copernicaanse Revolutie in de 16\textsuperscript{de} eeuw geen sprake.
        \end{itemize}
      \end{itemize}
      \item Waarom bleef de Copernicaanse Revolutie uit? 
      \begin{itemize}
        \item Revolutionibus werd beschouwd als een vrijblijvende wiskundige hypothese, zonder gevolgen in de realiteit. (Komt door voorwoord buiten Copernicus' weten om).
        \item Copernicus maakte zich zorgen voor spottende reacties $\Rightarrow$ die bleven uit omdat het Heliocentrisme werd genegeerd.
        \item 
      \end{itemize}
      \item OPGEPAST: Niet denken vanuit moderne karakter van het werk, en de tijdsgenoten de schuld geven, maar het standpunt van de tijdsgenoten innemen en het werk toetsen.
      \begin{itemize}
        \item Wetenschapper werkt juist in die tijd $\Rightarrow$ definieert project op basis van begrippen/opvattingen van die tijd.
        \item Om dit te onderzoeken moeten we dus ook die wetenschap uit die tijd begrijpen:
        \begin{itemize}
          \item In die tijd was Astronoom (wetenschap nu)=Astroloog (spekkers met de dierenriem): Astronomie stond enkel in teken van de astrologie.
          \item Copernicus kwam in contact met de ``Almagest'' (astronomisch standaardwerk). Dit gebruikte cirkels met aarde in midden. Copernicus plaatste de zon in het midden en dit vereenvoudigde alles. Bovendien kon hij hiermee de volgorde van de planeten bepalen.
          \item Revolutionibus: geen Astrologisch werk $\Rightarrow$ Werd wel beschouwd in hoe het tot debatten over Astrologie kon bijdragen. (Waren veel boeken die in debatten dienst deden.) $\Rightarrow$ Het boek ging alleen over Astronomie, dus niet zo nuttig in debatten. $\Rightarrow$ Astronomie werd overschaduwd door Astrologie.
          \item OPGELET: Bovenstaande steunt op Westman $\Rightarrow$ Kan zijn dat Copernicus zich probeerde te distanti\"eren van Astrologie.
        \end{itemize}
      \end{itemize}
      \item De Copernicaanse Revolutie bleef dus uit, maar het heliocentrisme moest wel ingang vinden in andere subcultuur (dan Astrologie/Astronomie), namelijk de natuurfilosofie.
    \end{itemize}

%%%%%%%%%%%%%%%%%%%%%%%%%%%%%%%%%%%%%%%%%%%%%%%%%%%%%%%%%%%%%%%%%%%%%%%%%%%%%%%%%%%%%%%%%%%%%%%%%%%%%%%%%%%%%%%%%%%%%%%%%%%%%%%%%%%%%%%%%%%%%%%%%%%%%%%%%%%%%%%%%%%%%%%%%%%%%%%%%%%%%%%%%%%%
  \newpage
  \section{Een wereld op papier}
  \begin{itemize}
    \item Niewigheid in 16\textsuperscript{de} eeuwse wetenschap: ILLUSTRATIES
    \begin{itemize}
      \item Maakte het mogelijk voor wetenschappers die geen toegang hadden tot dingen (zoals Neushoorns), om deze toch op een (min of meer) natuurgetrouwe manier te bekijken/te onderzoeken.
    \end{itemize}
    \item 16\textsuperscript{de} eeuwse natuurgetrouwe illustraties $\leftrightarrow$ Middeleeuwse schematische voorstellingen.
    \item Innovatie lijkt gevolg van empirische interesse
    \item Maar er was ook veel tegenstand
    \begin{itemize}
      \item Maakte directe observatie onnodig (waarom echt dissecteren als afbeelding toch essenti\"ele informatie bevat?)
      \item Vragen bij betrouwbaarheid van afbeeldingen (Vesalius' spiermannen waren idealen, geen afbeeldingen van dissecties, afbeeldingen van planten in verschillende groeistadia tegelijk)
      \item Kaarten konden niet gemakkelijk worden aangepast aan nieuwe ontdekkingen.
      \item Uitgevers betaalden voor afbeeldingen $\Rightarrow$ bepaalde wat in boek kwam (dus commerci\"ele redenen: monsters of mythologische figuren, die niks met tekst te maken hadden)
    \end{itemize}
    \item Afbeeldingen wijzen op verschuiving in het wetenschappelijk denken:
    \begin{itemize}
      \item Afbeelding toont particulier object $\leftrightarrow$ Universele begrippen (Aristoteles' essentie van de dingen)
    \end{itemize}
    \item Ook aandacht verschuift:
    \begin{itemize}
      \item Gericht op veelheid van planten en dieren, zonder poging om ordening/logica aan te brengen.
      \item Klein onderscheid tussen 2 planten kan van groot belang zijn $\leftrightarrow$ Universitaire ideaal (planten enkel voor geneeskunde) $\Rightarrow$ Natuurlijke historie vooral ontwikkeld buiten Universiteit, in netwerk van correspondenten.
      \item $\Rightarrow$ Nieuwe subcultuur van natuuronderzoekers: gedreven door praktische nieuwsgierigheid en niet door theoretische ambities.
    \end{itemize}
    \item Gebruik van afbeeldingen suggereert een tweede belangrijke verschuiving:
    \begin{itemize}
      \item Afbeeldingen vaak gebruikt in technische handleidingen, nodig om te begrijpen wat er in de tekst bedoeld wordt.
      \item $\Rightarrow$ Vervaging grens natuurlijk-artificieel: Afbeelding altijd artificieel, maar konden toch essentie vatten ($\leftrightarrow$ Aristoteles: artificieel (experimenten) zijn niet goed.)
    \end{itemize}
    \item Visuele cultuur is weerspiegeling van de grote interesse in natuurverschijnselen. (Zoals te zien in grote verzamelingen die niet wetenschappelijk zijn samengesteld, maar wel wetenschappers uitnodigde.)
  \end{itemize}

%%%%%%%%%%%%%%%%%%%%%%%%%%%%%%%%%%%%%%%%%%%%%%%%%%%%%%%%%%%%%%%%%%%%%%%%%%%%%%%%%%%%%%%%%%%%%%%%%%%%%%%%%%%%%%%%%%%%%%%%%%%%%%%%%%%%%%%%%%%%%%%%%%%%%%%%%%%%%%%%%%%%%%%%%%%%%%%%%%%%%%%%%%%%
  \newpage
  \section{Subculturen van wetenschap}
  \begin{itemize}
    \item 16\textsuperscript{de} eeuw:
    \begin{itemize}
      \item Ontstaan verschillende wetenschappelijke subculturen, gebaseerd op eigen visie natuurkennis.
      \item Onderscheiden van universiteiten. (Nauwelijk interesse in natuurwetenschap: Aristotelische natuurfilosofie in teken van academische afstudeerrichtingen (theologie, kerkelijk/burgelijk recht, geneeskunde))
      \item Universiteiten waren niet conservatief $\Rightarrow$ Open instellingen met informele discussies/idee\"enuitwisseling studenten-professoren.
      \item Universiteit: knooppunt internationaal verkeer idee\"en, boeken, objecten. Maar toch discours gebaseerd op Aristoteles' natuurfilosofie.
    \end{itemize}
    \item Subculteren:
    \begin{itemize}
      \item Humanisten (Desiderius Erasmus, Thomas More):
      \begin{itemize}
        \item Gebaseerd op filologische studie van teksten uit de Oudheid.
        \item Discussies over authenticiteit van overgeleverde teksten, interpretaties van Griekse/Latijnse uitdrukkingen.
        \item Uiteindelijke logische coherentie was GEEN doel! $\Rightarrow$ Enkel belangrijk: getrouwe weergave van de Ouden.
        \item Vorm en Stijl heel belangrijk!
        \item Toch brachten ze origineel werk (Vesalius weerlegde Gelanus terwijl hij er op steunde.)
      \end{itemize}
      \item Natuurlijke Historie:
      \begin{itemize}
        \item Studie van planten, dieren, en gesteenten.
        \item Ontstaan vanuit Humanisme.
        \item Probeerden planten die Dioscorides (Oude) had beschreven opnieuw te identificeren.
        \item Cre\"eerde internationaal netwerk van corresponderende Botanici.
        \item Ge\"ilustreerde kruidenboeken.
        \item Werd onderdeel van de elitaire cultuur van rijke verzamelaars.
        \item Cultuur verbrokkelde begin 17\textsuperscript{de} eeuw:
        \begin{itemize}
          \item Botanici voelden nood aan systematisering en classificatie.
          \item Verzamelaars hadde noog voor exotisch/esthetisch.
          \item Belangen van beide groepen (die samen Natuurhistorici vormden) liepen meer en meer uiteen.
          \item Het aantal te beschrijven soorten was ook gigantische toegenomen.
        \end{itemize}
      \end{itemize}
      \item Wonderboeken:
      \begin{itemize}
        \item Boeken waarin medische en technische recepten werden uitgelegd aan de ``gewone man'' in volkstaal.
        \item Commerci\"eel $\Rightarrow$ gaven goed aan wat publiek van wetenschap verwachtte.
        \item Gebaseerd op lokale ambachtslui.
        \item Auteurs waren meestal Humanisten.
        \item Gaf kritiek op academische wetenschap (boekverbranding) en riep op om kennis bij lokale ambachtsliederen te zoeken.
        \item Interesse voor Hermitisme:
        \begin{itemize}
          \item Verzamelnaam voor filosofisch-religieuze stromingen voor het Occulte.
          \item Inspiratie bij Alchemie, Astrologie, en Natuurlijke Magie.
        \end{itemize}
        \item Brede interesse voor wonderboeken $\Rightarrow$ Brede intersse voor nieuw wetenschappelijk kennisdomein: die van de ambachten.
        \item Introduceerde gebruik van doelbewuste experimenten met systematisch rapporteren.
        \item Nadruk op praktische kennis, met impliciete conclusie dat de natuur op zo'n manier verstaan moet worden.
        \item Moderen Wetenschap distantieerde zich achteraf wegens publicatie-cultuur en verwerping esoterische elementen van kennis over de natuur.
      \end{itemize}
      \item Wiskundige Cultuur:
      \begin{itemize}
        \item Kunstenaars (perspectiefleer) en ingenieurs (architectuur, cartografie,...)
        \item In Itali\"e
        \item Situeert zich tussen Universitair (Astrologie/Astronomie) en praktische reken- en meetkunde.
        \item Wiskunde als toe-eigening van status: door wiskunde kreeg kunst een theoretische legitimatie.
        \item Kunstenaars vooral tewerkgesteld in adellijke hoven als hofwiskundigen.
        \item Nadruk op uistallen van pracht, rijkdom, en status.
        \item Bracht Galilei voort.
        \item Ook ten Noorden van de Alpen: andere achtergrond:
        \begin{itemize}
          \item Nadruk op praktisch nut van ingenieurs in dienst van handelaars/overheden (Engeland: Navigatie -Nederland: Inpoldering/Vestingbouw).
        \end{itemize} 
      \end{itemize}
    \end{itemize}
  \end{itemize}

%%%%%%%%%%%%%%%%%%%%%%%%%%%%%%%%%%%%%%%%%%%%%%%%%%%%%%%%%%%%%%%%%%%%%%%%%%%%%%%%%%%%%%%%%%%%%%%%%%%%%%%%%%%%%%%%%%%%%%%%%%%%%%%%%%%%%%%%%%%%%%%%%%%%%%%%%%%%%%%%%%%%%%%%%%%%%%%%%%%%%%%%%%%%
  \newpage
  \section{De wiskundige natuurfilosofie van Galilei}

%%%%%%%%%%%%%%%%%%%%%%%%%%%%%%%%%%%%%%%%%%%%%%%%%%%%%%%%%%%%%%%%%%%%%%%%%%%%%%%%%%%%%%%%%%%%%%%%%%%%%%%%%%%%%%%%%%%%%%%%%%%%%%%%%%%%%%%%%%%%%%%%%%%%%%%%%%%%%%%%%%%%%%%%%%%%%%%%%%%%%%%%%%%%
  \newpage
  \section{Een nieuwe kosmologie}

%%%%%%%%%%%%%%%%%%%%%%%%%%%%%%%%%%%%%%%%%%%%%%%%%%%%%%%%%%%%%%%%%%%%%%%%%%%%%%%%%%%%%%%%%%%%%%%%%%%%%%%%%%%%%%%%%%%%%%%%%%%%%%%%%%%%%%%%%%%%%%%%%%%%%%%%%%%%%%%%%%%%%%%%%%%%%%%%%%%%%%%%%%%%
  \newpage
  \section{Nieuwsgierigheid en materi\"ele cultuur}

%%%%%%%%%%%%%%%%%%%%%%%%%%%%%%%%%%%%%%%%%%%%%%%%%%%%%%%%%%%%%%%%%%%%%%%%%%%%%%%%%%%%%%%%%%%%%%%%%%%%%%%%%%%%%%%%%%%%%%%%%%%%%%%%%%%%%%%%%%%%%%%%%%%%%%%%%%%%%%%%%%%%%%%%%%%%%%%%%%%%%%%%%%%%
  \newpage
  \section{Newton en de Wetenschappelijke Revolutie}

%%%%%%%%%%%%%%%%%%%%%%%%%%%%%%%%%%%%%%%%%%%%%%%%%%%%%%%%%%%%%%%%%%%%%%%%%%%%%%%%%%%%%%%%%%%%%%%%%%%%%%%%%%%%%%%%%%%%%%%%%%%%%%%%%%%%%%%%%%%%%%%%%%%%%%%%%%%%%%%%%%%%%%%%%%%%%%%%%%%%%%%%%%%%
  \newpage
  \section{Iedereen Newtoniaan!}

%%%%%%%%%%%%%%%%%%%%%%%%%%%%%%%%%%%%%%%%%%%%%%%%%%%%%%%%%%%%%%%%%%%%%%%%%%%%%%%%%%%%%%%%%%%%%%%%%%%%%%%%%%%%%%%%%%%%%%%%%%%%%%%%%%%%%%%%%%%%%%%%%%%%%%%%%%%%%%%%%%%%%%%%%%%%%%%%%%%%%%%%%%%%
  \newpage
  \section{Nuttige wetenschap}

%%%%%%%%%%%%%%%%%%%%%%%%%%%%%%%%%%%%%%%%%%%%%%%%%%%%%%%%%%%%%%%%%%%%%%%%%%%%%%%%%%%%%%%%%%%%%%%%%%%%%%%%%%%%%%%%%%%%%%%%%%%%%%%%%%%%%%%%%%%%%%%%%%%%%%%%%%%%%%%%%%%%%%%%%%%%%%%%%%%%%%%%%%%%
  \newpage
  \section{Gescheiden wegen}

%%%%%%%%%%%%%%%%%%%%%%%%%%%%%%%%%%%%%%%%%%%%%%%%%%%%%%%%%%%%%%%%%%%%%%%%%%%%%%%%%%%%%%%%%%%%%%%%%%%%%%%%%%%%%%%%%%%%%%%%%%%%%%%%%%%%%%%%%%%%%%%%%%%%%%%%%%%%%%%%%%%%%%%%%%%%%%%%%%%%%%%%%%%%
  \newpage
  \section{Wetenschap aan de universiteit}


\end{document}
